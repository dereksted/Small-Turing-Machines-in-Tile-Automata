\section{Tile Automata}\label{def:tile-automata}

The Tile Automata model is a model that combines elements from cellular automata and the 2-Handed Assembly Model (2HAM).  Like 2HAM, the fundamental unit of Tile Automata is the Tile.  Tiles are a unit square with a state and a position centered on a point on the square lattice over the integers in two dimensions, so that a tile's coordinates (x, y) $\in \mathbb{Z}$.  The affinity function of Tile Automata consists of a set of two states with either an above-below orientation (denoted as $\bot$) or side-side orientation (denoted as $\vdash$) and an attachment strength. Tiles can bond with other tiles by attaching to one another according to the affinity function.  For this bonding to occur and persist, there must be the condition of $\tau$ stability.  For there to be $\tau$ stability, the attachment strength between tiles or a group of tiles must be equal to or greater then the stability threshold or $\tau$ of the Tile Automata system.  An assembly is a $\tau$ stable connected set of Tile Automata tiles such that there exists no way to seperate the tiles with breaking bondds of at least $\tau$ strength.  Finally, from cellular automata, Tile Automata takes the concept of state changes.  Transitions occur in Tile Automata according to a transition rule.  A transition rule consists of an input of a pair of adjacent tiles and an output of new states for the tiles.  One or both tiles can change state.  